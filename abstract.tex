\section{El coste de la abstracción}

\begin{frame}[t]{Todo el mundo sabe que...}
\begin{itemize}
  \pause
  \item C++ es más lento que C.
    \begin{itemize}
      \item Debe serlo puesto que es de más alto nivel.
    \end{itemize}

  \vfill\pause
  \item La abstracción siempre tiene un precio en tiempo de ejecución.
    \begin{itemize}
      \item Así que no hay nada como el código de bajo nivel.
      \item Si puede ser en ensamblador.
    \end{itemize}

  \vfill\pause
  \item El mejor rendimiento se obtiene con optimizaciones de bajo nivel.
    \begin{itemize}
      \item Cuanto más pegado al hardware mejor.
    \end{itemize}

  \vfill\pause
  \item Cada enfoque de resolución de un problema necesita una versión distinta
        del código.
    \begin{itemize}
      \item No va a ser lo mismo una versión secuencial que una con OpenMP.
    \end{itemize}
\end{itemize}
\end{frame}

\begin{frame}[t]{Caso práctico: Fludianimate}
\begin{itemize}
  \item Parte del benchmark PARSEC.
    \begin{itemize}
      \item \url{http://parsec.cs.princeton.edu}
      \item Originalmente desarrollado por Intel.
      \item Versiones en simple y doble precisión.
      \item Versiones secuencial, PThreads, Intel TBB.
    \end{itemize}

  \vfill\pause
  \item Características:
    \begin{itemize}
      \item en su mayoría código tipo-C.
      \item Uso de preprocesador.
      \item Funciones sin paso de parámetros accediendo a globales.
      \item Pool de memoria ajustado a caché de procesador.
    \end{itemize}
  
\end{itemize}
\end{frame}

\begin{frame}{Métricas}
\begin{tabular}{|l|r|r|}
\hline
Métrica & Original & Refactorizado\\
\hline
\hline
Lines		& 1873	& 1330\\
Effective LOC 	& 1093	& 846\\
Logical LOC 	& 880	& 449 \\
\hline
Functions	& 87	& 125\\
Max Cyclomatic Complexity	& 33	& 8\\
Average Cycolomatic Complexity	& 3.39	& 1.35\\
\hline
Classes		& 6	& 16 \\
Max Cyclomatic Complexity	& 65 & 45\\
Average Cyclomatic Complexity	& 11.00 & 9.12\\
\hline
\end{tabular}
\end{frame}

\begin{frame}{Tiempo de ejecución por iteración}
\begin{tikzpicture}
\begin{semilogyaxis}[xlabel=Iterations,ylabel=Normalized Execution time (us), legend pos=south east]
\addplot table[header=false]{perf-fluid/p2000/fanimate.seq.norm.dat};
\addlegendentry{original}
\addplot table[header=false]{perf-fluid/p2000/animate.seq.norm.dat};
\addlegendentry{refactored}
\end{semilogyaxis}
\end{tikzpicture}
\end{frame}

\begin{frame}{Tiempo de ejecución por iteración (4 cores, 8 threads)}
\begin{tikzpicture}
\begin{semilogyaxis}[xlabel=Iterations,ylabel=Normalized Execution time (us), legend pos=south east]
\addplot table[header=false]{perf-fluid/p500/fanimate.tbb.8.norm.dat};
\addlegendentry{original}
\addplot table[header=false]{perf-fluid/p500/animate.tbb.8.norm.dat};
\addlegendentry{refactored}
\end{semilogyaxis}
\end{tikzpicture}
\end{frame}

\begin{frame}[t]{Para saber más}
\begin{itemize}
  \item \textmark{Improving performance and maintainability through refactoring in C++11}.
        J.D. Garcia y B. Stroustrup.
        ISO C++ Foundation, 2015.
        \url{https://isocpp.org/blog/2015/10/garcia-stroustrup-refactoring}.
\end{itemize}
\end{frame}
