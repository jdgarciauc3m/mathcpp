\subsection{Números aleatorios}

\begin{frame}[t]{Motores de generación}
\begin{itemize}
  \item Generadores pseudoaleatorios:
    \begin{itemize}
      \item \cppid{linear\_congruential\_engine}.
      \item \cppid{mersenne\_twister\_engine}.
      \item \cppid{subtract\_with\_carry\_engine}.
    \end{itemize}
  \pause\vfill
  \item Adaptdores de motores:
    \begin{itemize}
      \item \cppid{discard\_block\_engine}.
      \item \cppid{independent\_bits\_engine}.
      \item \cppid{shuffle\_order\_engine}.
    \end{itemize}
  \pause\vfill
  \item Generadores no deterministas:
    \begin{itemize}
      \item \cppid{randon\_device}.
    \end{itemize}
\end{itemize}
\end{frame}

\begin{frame}[t]{Motores de generación}
\begin{itemize}
  \item Generan valores enteros uniformes de algún tipo entero.
    \begin{itemize}
      \item Basado en una semilla.
      \item Distintos compromisos entre velocidad y almacenamiento.
      \item Muy genéricos
        \begin{itemize}
          \item Requiren parametrización cuidadosa.
        \end{itemize}
    \end{itemize}
  \vfill\pause
  \item Casos más comunes predefinidos:
    \begin{itemize}
      \item \cppid{minstd\_rand0}, \cppid{minstd\_rand} $\Rightarrow$ Congruencial lineal.
        \begin{itemize}
          \item Lewis, Goodman, Miller (1969).
        \end{itemize}
      \item \cppid{mt19937}, \cppid{mt19937\_64} $\Rightarrow$ Mersenne-Twister.
        \begin{itemize}
          \item Matsumoto, Nishimura (1998, 2000).
        \end{itemize}
      \item \cppid{ranlux24}, \cppid{ranlux48}.
        \begin{itemize}
          \item L\"{u}scher, James (1994).
        \end{itemize}
      \item \cppid{knuth\_b}
    \end{itemize}
\end{itemize}
\end{frame}

\begin{frame}[t]{Distribuciones}
\begin{itemize}
  \item Distribuciones uniformes.
    \begin{itemize}
      \item \cppid{uniform\_int\_distribution}, \cppid{uniform\_real\_distribution}.
    \end{itemize}
  \vfill\pause
  \item Distribuciones de Bernoulli.
    \begin{itemize}
      \item
        \cppid{bernoulli\_distribution},
        \cppid{binomial\_distribution},
        \cppid{negative\_binomial\_distribution},
        \cppid{geometric\_distribution}.
    \end{itemize}
  \vfill\pause
  \item Distribuciones de Poisson.
    \begin{itemize}
      \item
        \cppid{poisson\_distribution},
        \cppid{exponential\_distribution},
        \cppid{gamma\_distribution},
        \cppid{weibull\_distribution},
        \cppid{extreme\_value\_distribution}.
    \end{itemize}
  \vfill\pause
  \item Distribuicones normales.
    \begin{itemize}
      \item
        \cppid{normal\_distribution},
        \cppid{log\_normal\_distribution},
        \cppid{chi\_squared\_distribution},
        \cppid{cauchy\_distribution},
        \cppid{fisher\_f\_distribution},
        \cppid{student\_t\_distribution}.
    \end{itemize}
\end{itemize}
\end{frame}

\begin{frame}[t,fragile]
\begin{block}{Múltiples fuentes aleatorias}
\begin{lstlisting}
int main() {
  using namespace std;
  seed_seq seq{1,2,3,4};
  vector<uint32_t> seeds{10};
  seq.generate(sees.begin(), seeds.end());
  
  vector<mt19935_64> engines;
  for (int i=0; i<10; ++i) {
    engines.emplace_back(engines[i]);
  }

  normal_distribution dist{25.0, 1.5};
  for (int i=0; i<max; ++i) {
    for (auto & e : engines) {
      cout << "valor " << i << " , dist " << j << ":"
           << dist(e[j]) << "\n";
    }
  }
}
\end{lstlisting}
\end{block}
\end{frame}
