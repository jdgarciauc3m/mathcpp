\subsection{Funciones matemáticas especiales}

\begin{frame}[t]{Funciones matemáticas especiales}
\begin{itemize}
  \item Origen:
    \begin{itemize}
      \item \textgood{2007}: Informe técnico \textmark{ISO/IEC TR 17968:2007}.
      \item \textgood{2010}: Estándar internacional \textmark{ISO/IEC 29124:2010}.
      \item \textgood{2017}: Incorporadas a C++17 \textmark{ISO/IEC 14882:2017}.
    \end{itemize}

  \vfill\pause
  \item Funciones:
    \begin{itemize}
      \item Leguerre: Polinomio asociado, polinomio de Leguerre.
      \item Legendre: Polinomio asociado, polinomio de Legendre.
      \item Hermite: Polinomio de Hermite.
      \item Funciones elípticas: Primera, segunda y tercera integral elíptica.
        \begin{itemize}
          \item Completas e incompletas.
        \end{itemize}
      \item Funciones de Bessel: Regular, irregular, de primer orden.
        \begin{itemize}
          \item Cilíndricas y esféricas.
        \end{itemize}
      \item Integral exponencial.
      \item Otras funciones: Función beta, función z de Rieman.. 
    \end{itemize}
  
\end{itemize}
\end{frame}

\begin{frame}[t,fragile]{Ejemplo}
\begin{block}{Polinomios}
\begin{lstlisting}
int main() {
  using namespace std;

  cout << "Hermite3(10)" << hermite(3,10) << "\n";
  cout << "Legendre3(0.25)" << legendre(3,0.25) << "\n";
  cout << "Leguerre3(0.5)" << leguerre(3,0.25) << "\n";
}
\end{lstlisting}
\end{block}
\end{frame}
