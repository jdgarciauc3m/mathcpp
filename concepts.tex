\section{Conceptos}

\begin{frame}[t]{Gran éxito de la programación genérica}
\begin{itemize}
  \item Algunas razones:
    \begin{itemize}
      \item Flexibilidad: Las plantillas son un sub-lenguaje \textmark{Turing-complete}.

      \vfill\pause
      \item Seguridad en tipos: Detección de inconsistencias en tiempo de compilación.

      \vfill\pause
      \item Mecanismos de especialización.
        \begin{itemize}
          \item Tipos irregulares.
          \item Metaprogramación genérica.
          \item Optimizaciones basadas en tipos.
        \end{itemize}

      \vfill\pause
      \item Rendimiento superior.
    \end{itemize}
\end{itemize}
\end{frame}

\begin{frame}[t]{Pero...}
\begin{itemize}
  \item También tiene grandes inconvenientes
    \begin{itemize}
      \vfill\item Sintaxis verbosa.
      \vfill\item \emph{Duck-typing}.
      \vfill\item Mensajes de error incomprensibles.
      \vfill\item Sobrecarga confusa.
      \vfill\item Problemas de organización de código.
      \vfill\item Compilación lenta.
    \end{itemize}
\end{itemize}
\end{frame}

\begin{frame}[t,fragile]
\begin{block}{Sumando de forma genérica}
\begin{lstlisting}
template <typename T>
T suma(const vector<T> & v) {
  T r{};
  for (auto i = v.begin(); i!=v.end(); ++i) {
    r+= *i;
  }
}

void f() {
  std::vector<double> v { 1.0, 2.0, 3.0 };
  std::cout << suma(v) << "\n";

  std::vector<complex<float>> w = readvalues();
  std::cout << suma(v) << "\n";

  std::vector<std::string> u {"C++", "mola"};
  std::cout << suma(u) << "\n"; // un no es un vector de números???
}
\end{lstlisting}
\end{block}
\end{frame}

\begin{frame}[t]{Tipos y conceptos}
\begin{itemize}
  \item \textgood{Tipo}:
    \begin{itemize}
      \item Especifica el conjunto de valores que puede tomar una entidad.
      \item Define el conjunto de operaciones que son aplicables a sus valores.
      \item Especifica la representación en memoria.
    \end{itemize}

  \vfill\pause
  \item \textgood{Concepto}:
    \begin{itemize}
      \item Especifica el conjunto de tipos que satisfacen el concepto.
      \item Define el conjunto de operaciones que son aplicables a sus valores.
      \item \textbad{No} especifica la representación en memoria.
    \end{itemize}
\end{itemize}
\end{frame}

\begin{frame}[t,fragile]
\begin{block}{Sumando de forma genérica}
\begin{lstlisting}
template <typename N>
  requires Number<N>
N suma(const vector<N> & v) {
  N r{};
  for (auto i = v.begin(); i!=v.end(); ++i) {
    r+= *i;
  }
}

void f() {
  std::vector<double> v { 1.0, 2.0, 3.0 };
  std::cout << suma(v) << "\n";

  std::vector<complex<float>> w = readvalues();
  std::cout << suma(v) << "\n";

  std::vector<std::string> u {"C++", "mola"};
  std::cout << suma(u) << "\n"; // un no es un vector de números???
}
\end{lstlisting}
\end{block}
\end{frame}

\begin{frame}[t,fragile]
\begin{block}{Sumando de forma genérica}
\begin{lstlisting}
template <Number N>
N suma(const vector<N> & v) {
  N r{};
  for (auto i = v.begin(); i!=v.end(); ++i) {
    r+= *i;
  }
}

void f() {
  std::vector<double> v { 1.0, 2.0, 3.0 };
  std::cout << suma(v) << "\n";

  std::vector<complex<float>> w = readvalues();
  std::cout << suma(v) << "\n";

  std::vector<std::string> u {"C++", "mola"};
  std::cout << suma(u) << "\n"; // un no es un vector de números???
}
\end{lstlisting}
\end{block}
\end{frame}

\begin{frame}[t,fragile]{Definiendo conceptos}
\begin{block}{El concepto \emph{Number}}
\begin{lstlisting}
template <typename N>
concept Number =
  requires (N x, N y) {
    { N{} };
    { N{x} };
    { x += y } -> N&;
  };
\end{lstlisting}
\end{block}
\end{frame}

\begin{frame}[t]{Para saber más}
\begin{itemize}
  \item ISO/IEC TS 19217:2015: \textmark{C++ Extensions for Concepts}.
        Technical Specification.
        ISO.

  \vfill
  \item \textmark{Concepts: The future of generic programmiing (the future is here)}.
        Bjarne Stroustrup.
        \url{https://youtu.be/PU-2ntDuF10}
\end{itemize}
\end{frame}
