\section{Programación basada en contratos}

\begin{frame}[t]{Corrección frente a robustez}
\begin{itemize}
  \item En el diseño de un componente hay una tensión entre
        dos propiedades \textmark{relacionadas} y \textmark{ortogonales}:
    \begin{itemize}
      \item \textgood{robustez}.
      \item \textgood{corrección}.
    \end{itemize} 

  \vfill\pause
  \item \textgood{Robustez}: Capacidad de un componente de reaccionar
        de forma apropiada a condiciones anormales.

  \vfill\pause
  \item \textgood{Corrección}: Grado en el que un componente software
        cumple con su especificación.
\end{itemize}
\end{frame}

\begin{frame}[t]{Programación basada en contratos}
\begin{itemize}
  \item Basada originalmente en las ideas de la \textmark{lógica de Hoare}.
    \begin{itemize}
      \item Conceptos de \textgood{precondición} y \textgood{postcondición}.
    \end{itemize}
  \vfill\pause
  \item Desarrollado posteriormente por Bertrand Meyer en el lenguaje Eiffel.
  \vfill
  \item Metáfora de componentes como \textgood{cliente} y \textgood{servidor}
        que se relacionan mediante un \textmark{contrato} con 
        \textmark{obligaciones} y \textmark{beneficios}.
    \begin{itemize}
      \item\pause Las \textgood{precondiciones} son una 
          \textmark{obligación} para el \textgood{cliente} y un
          \textmark{beneficio} para el \textgood{suministrador}.
      \item\pause Las \textgood{postcondiciones} son una
          \textmark{obligación} para el \textgood{suministrador} y un
          \textgood{beneficio} para el \textmark{cliente}.
    \end{itemize}
\end{itemize}
\end{frame}

\begin{frame}[t,fragile]{Comprobaciones}
\begin{itemize}
  \item Una \textmark{comprobación} es un predicado que forma parte
        de un contrato y que se comprueba en tiempo de ejecución
        para determinar si el software tiene un defecto.
\end{itemize}
\begin{block}{Contrato para una pila}
\begin{lstlisting}
template <typename T>
class stack {
public:
  stack() [[ensures: size() == 0]];

  int size() [[ensures r: r>=0]];

  void push(const T & x)
    [[expects: !is_full()]]
    [[ensures: !is_empty()]];
  //...
}
\end{lstlisting}
\end{block}
\end{frame}

\begin{frame}[t,fragile]{Axiomas}
\begin{itemize}
  \item Un \textmark{Axioma} es un postulado que se tiene por cierto.
    \begin{itemize}
      \item Se pueden utilizar en el análisis de los programas para
            determinar inconsistencias.
      \item Se pueden utilizar durante la optimización del programa.
      \item Permiten el uso de predicados que no son \emph{idempotentes}.
    \end{itemize}
\end{itemize}
\begin{block}{Axiomas de la búsqueda binaria}
\begin{lstlisting}
template <typename Iter, typename T>
Iter busqueda_binaria(Iter primero, Iter ultimo, T value)
  [[expect axiom: se_puede_llegar(primero,ultimo)]]
  [[expect audit: esta_ordenado(primero,ultimo)]]
  [[ensures axiom res: se_puede_llegar(primero,res) &&
                       se_puede_llegar(res,ultimo)]];
\end{lstlisting}
\end{block}
\end{frame}

\begin{frame}[t]{Para saber más}
\begin{itemize}
  \item \textmark{An axiomatic basis for computer programming}.
        C.A.R. Hoare.
        Communications of the ACM, 12(1):576--580.
        1969.

  \vfill
  \item \textmark{Object Oriented Software Construction}.
        Bertrand Meyer.
        Prentice-Hall, 1997.

  \vfill
  \item P0542R5: \textmark{Support for contract based programming in C++}
        G. Dos Reis, J.D. Garcia, J. Lakos, A. Meredith, N. Myers, B. Stroustrup.
        ISO/IEC JTC1/SC22/WG21 working paper. Junio 2018.
        \url{http://wg21.link/p0542r5}.
\end{itemize}
\end{frame}
