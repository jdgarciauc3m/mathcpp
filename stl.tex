\subsection{Contenedores, iteradores, algoritmos}

\begin{frame}[t]{STL: Standard Template Library}
\begin{itemize}
  \item Alex Stepanov.
    \begin{itemize}
      \item Abstracción a partir de algoritmos concretos y eficientes.
      \item Obtención de algoritmos que se pueden usar con distintas represetaciones
            de datos.
      \item Una de las claves del éxito de C++.
    \end{itemize}

  \vfill\pause
  \item Elementos de STL:
    \begin{itemize}
      \item Diversidad de contenedores.
      \item Iteradores sobre contenederes.
      \item Algoritmos independientes de contenedores expresados en
            términos de iteradores.
    \end{itemize}
\end{itemize}
\end{frame}

\begin{frame}[t,fragile]{¿Qué es un contenedor?}
\begin{itemize}
  \item Un \textgood{contenedor} es una estructura que puede
        \textmark{almacenar valores}.
    \begin{itemize}
      \item En STL los contenedores son homogéneos.
    \end{itemize}

  \vfill\pause
  \item Tipos principales de contenedores:
    \begin{itemize}
      \item \textmark{Contenedores de secuencia}: Almacenan una secuencia de valores.
      \item \textmark{Contenedores asociativos}: Almacenan pares \emph{clave-valor}.
    \end{itemize}
\end{itemize}
\end{frame}

\begin{frame}[t,fragile]{Contenedores de secuencia}
\begin{itemize}
  \item Almacenan una secuencia de valores de un determinando tipo.
  \item Pueden permitir el acceso mediante operaciones propias o
        a través de iteradores.
  \item Varios compromisos: \cppid{vector}, \cppid{list}, \cppid{deque}, \cppid{forward\_list}.
\end{itemize}
\begin{block}{Reduciendo un vector}
\begin{lstlisting}
double suma(std::vector<double> & v) {
  double r = 0.0;
  for (size_t i=0; i<v.size(); ++i) {
    r += v[i];
  }
  return r;
}
\end{lstlisting}
\end{block}
\end{frame}

\begin{frame}[t,fragile]{Iteradores}
\begin{itemize}
  \item Un \textgood{iterador} es un valor que permite recorrer los elementos
        de un contenedor.
    \begin{itemize}
      \item Cada contenedor tiene un tipo de iterador asociado.
    \end{itemize}

  \vfill\pause
  \item Operaciones mínimas:
    \begin{itemize}
      \item Acceder al objeto apuntado (\cppid{*i}).
      \item Avanzar al siguiente elemento (\cppid{++i}).
      \item Comparar si apuntan a la misma posición (\cppid{i!=j}).
    \end{itemize}

  \vfill\pause
  \item Posiciones especiales:
    \begin{itemize}
      \item \textgood{Inicio}: Primera posición del contenedor.
      \item \textgood{Fin}: Siguiente a la última posición del contenedor.
    \end{itemize}
  \vfill
  \item \textmark{Observación}: Cualquier subsecuencia se puede representar
        como un intervalo semiabierto \cppid{[i,j)}.
\end{itemize}
\end{frame}

\begin{frame}[t,fragile]{Uso de iteradores}
\begin{block}{Reducción de un vector}
\begin{lstlisting}[escapechar=@]
double suma1(std::vector<double> & v) {
  double r = 0.0;
  for (typename std::vector<double>::iterator i = v.begin(); i!=v.end(); ++i) {
    r += *i;
  }
  return r;
}
@\pause@
double suma2(std::vector<double> & v) {
  double r = 0.0;
  for (auto i = v.begin(); i!=v.end(); ++i) {
    r += *i;
  }
  return r;
}
\end{lstlisting}
\end{block}
\end{frame}

\begin{frame}[t,fragile]{Algoritmos}
\begin{itemize}
  \item Un \textgood{algoritmo} sobre una secuencia de datos se puede
        definir en términos de los iteradores que definen la secuencia.
    \begin{itemize}
      \item Esto independiza los algoritmos de las estructuras de datos concretas.
      \item Hace los algoritmos válidos para estructuras de datos que 
            todavía no existen.
    \end{itemize}
\end{itemize}
\vfill\pause
\begin{block}{Reducción de un vector}
\begin{lstlisting}
double suma(const std::vector<double> & v) {
  return std::reduce(v.begin(), v.end());
}
\end{lstlisting}
\end{block}
\end{frame}
